\label{chapter:consideracoes}

Este trabalho abordou o desenvolvimento de um sistema computacional que consiste de uma boia que utiliza um computador de placa única para detecção de objetos (utilizando aprendizado de máquina) dentro de ambientes fluviais com alta turbidez, para auxiliar o trabalho de mergulhadores, pesquisadores e profissionais que utilizam águas fluviais como ambiente de trabalho.

Ao longo do desenvolvimento do projeto foram observados trabalhos semelhantes que ajudaram a construir um método proposto (sistema computacional) para solucionar o problema proposto nesse trabalho de conclusão de curso.
% O fluxo da boia foi projetado, com o desenvolvimento de um fluxograma do funcionamento do sistema proposto, com isso foi desenvolvido um diagrama de sequência afim de validar os passos a serem tomados durante o tempo de vida de cada processo, futuramente esse diagrama será utilizado para geração de uma rede de Petri.

Parte do objetivo do sistema computacional proposto é especular o tamanho dos objetos detectados e relatar para o usuário. Essa é uma atividade ainda em desenvolvimento e pesquisa, assim este primeiro momento teve como foco, o projeto de um método para identificar e classificar os objetos dentro da água.

A Boia possui grande potencial para adição de sensores no futuros, sensores esses como um oxímetro, ou motores para locomoção e estabilização. A Boia será projetada para suportar tais modificações de forma com que os sensores sejam acoplados com facilidade, evitando desgaste do no corpo da Boia.