\label{chapter:consideracoes}

Este trabalho abordou o desenvolvimento de uma boia utilizado um computador de placa única para detecção de objetos dentro de ambientes fluviais com alta turbidez, utilizando aprendizado de máquina para auxiliar o trabalho de mergulhadores, pesquisadores e profissionais que utilizam águas fluviais como ferramenta de trabalho.

Ao longo do desenvolvimento do projeto foram observados trabalhos semelhantes com problemas similares, tais trabalhos ajudaram a construir um método para solucionar o problema do sistema computacional proposto nesse trabalho de conclusão de curso.

O fluxo da boia foi projetado, com o desenvolvimento de um fluxograma do funcionamento do sistema proposto, com isso foi desenvolvido um diagrama de sequência afim de validar os passos a serem tomados durante o tempo de vida de cada processo, futuramente esse diagrama será utilizado para geração de uma rede de Petri.

Parte do objetivo do sistema computacional proposto é especular o tamanho dos objetos detectados e relatar para o usuário. Essa será uma implementação futura uma vez que o sistema tem foco em identificar e classificar os objetos dentro da água.

A Boia possui grande potencial para adição de sensores no futuros, sensores esses como um oxímetro, ou motores para locomoção e estabilização. A Boia será projetada para suportar tais modificações de forma com que os sensores sejam acoplados com facilidade, evitando desgaste do no corpo da Boia.