
No Brasil há uma grande diversidade de redes fluviais que são divididas em regiões hidrográficas. Uma dessas regiões é a Bacia Amazônica, considerada a mais extensa do mundo \cite{portalbrasil:2009}. As águas da Bacia Amazônica são classificadas em branca, preta e clara \cite{sioli:2012}. Segundo \citeonline{barthem:2004} os rios de água branca tem turbidez elevada dificultando a visibilidade dentro da água. De acordo com \citeonline{santos:2005} essa coloração branca é causada pela riqueza de minerais na água, e frequentemente encontrada nos rios da região norte do Brasil.


A demanda pela identificação de objetos submersos tem crescido, dado ao grande desenvolvimento nas observações oceânicas, que resultam em grandes quantidades de dados visuais de ambientes submersos a serem processados. Reconhecimento de espécies de de peixes e resolver os problemas na detecção de objetos são as tarefas mais importantes na observação oceânica, beneficiando cientistas e biólogos marinhos, bem como aplicações comercias na área da piscicultura~\cite{Qin:2016}. 


Em aplicações para identificação de peixes, câmeras colocadas em redes de observação oceânicas enfrentam dificuldades extremas causadas pelos ambientes naturais, tal como atenuação da luz e recifes de corais. Segundo \citeonline{Qin:2016}, com a utilização de matriz de decomposição para processar os vídeos, é possível extrair as linhas dos peixes em primeiro plano, assim eliminando o fundo da imagem, facilitando o processo de reconhecimento dos peixes.


Segundo \citeonline{vandamme:2015} é possível fazer a captação de dados em ambientes submersos, utilizando técnicas elementares como desenhos em escala, trilateração de fita métrica, medidas de deslocamento e fotografia simples. Esses métodos são ideais para fazer reconhecimento de sítios arqueológicos submersos. Porém esses métodos não são precisos, com exceção da fotografia, e levam muito tempo para serem construídos, e são propensos a erros humanos. Essas técnicas produzem apenas representações bidimensionais e tridimensionais com baixo nível de detalhes do local estudado.

Segundo \citeonline{Lu:2017}, o som pode ser usado para mapear ambientes, emitindo um pulso que reflete no fundo do oceano criando um sonograma, que é a representação gráfica através de frequências de som. As imagens obtidas por este sonar se assemelham a imagens óticas, com níveis de detalhes bem superiores. O reflexo criado por esse sonar tem formato de leque, com a medida que o pulso se movimenta, os reflexos irão criar séries de linhas de imagem, perpendiculares ao feixe. Dependendo do ambiente estudado, o sonograma pode ser confuso, sendo necessário ter muita experiência para identificar as imagens.

Na análise feita por \citeonline{vandamme:2015} é afirmado que há técnicas avançadas de coleta tridimensional submersa, que são eficientes e altamente precisas. De acordo \citeonline{watson:2005}, uma dessas técnicas é o remote \textit{stereo-video technique}, que consiste em duas câmeras controladas remotamente no fundo do oceano. \citeonline{vandamme:2015} também descreve outra técnica como \textit{Computer Vision Photogrammetry} (fotogrametria de visão computacional), que permite que uma série de imagens sejam carregadas em um software dedicado para gerar uma modelo tridimensional da cena ou objeto.


Visando contribuir com o desenvolvimento de sistemas computacionais para processamento de imagens, o contexto deste trabalho está situado em projetar e desenvolver um sistema computacional móvel que seja capaz de detectar objetos submersos em águas turvas, utilizando: uma câmera ótica submersa, visão computacional e algoritmos de classificação de padrões. Assim, os dados colhidos através do sistema serão enviados para um computador ou aplicativo móvel através de uma conexão sem fio, e então serão processados e classificados. O sistema será controlado a partir de um computador e irá fazer uma estimativa da distância e dimensões do objeto de acordo com a classificação. O sistema irá facilitar o trabalho de identificação e detecção de objetos submersos.


\section{Definição do problema} 

%\textcolor{red}{TEXTO INTRODUTORIO\todo{Adicionar um breve texto para motivar a sua questão de pesquisa} Texto:  por exemplo, em um rio com baixa visibilidade que contém destroços pode interferir na segurança e eficiência no trabalho de um mergulhador.}
%Como projetar um sistema computacional capaz de detectar e classificar objetos submersos em ambiente aquáticos com alta turbidez e parcialmente observáveis, de forma que a distância e as dimensões dos objetos sejam estimadas, e adicionalmente que os usuários deste sistema possam opera-lo remotamente?

Com a dificuldade na locomoção e exploração dentro de ambientes aquáticos com turbidez comprometendo o desempenho de profissionais e pesquisadores, é necessário um método para auxiliar a visibilidade e detecção de possíveis destroços e objetos nocivos. Por exemplo se um rio com baixa visibilidade contém destroços, pode interferir na segurança e eficiência no trabalho de um mergulhador
%

Neste sentido, o problema considerado neste trabalho é expresso na seguinte questão: \textbf{Como projetar um sistema computacional para identificar ou detectar objetos submersos em ambiente aquáticos com alta turbidez e parcialmente observáveis, de forma que a distância e as dimensões do objetos sejam estimados?} 

% 

%\section{Motivação} 


\section{Objetivo Geral}

O objetivo principal deste trabalho é projetar e avaliar um sistema computacional móvel que seja capaz de detectar objetos submersos em ambientes aquáticos com baixa visibilidade, e que possa calcular a estimativa da distância e as dimensões de objetos baseado em processamento de imagens, 
% utilizando técnicas de visão computacional, 
de tal forma que o sistema proposto possa ser aplicado para auxiliar em missões de exploração ou de resgate em ambientes aquáticos, provendo um modo de superar as limitações da visão ou atuação humana.

\section{Objetivos Específicos}
Os objetivos específicos são os seguintes:
\begin{enumerate}
	\item Identificar métodos para a modelagem do sistema proposto em termos de software e do hardware;

	\item Definir um modelo formal que represente o fluxo de execução do sistema proposto, visando analisar propriedades de segurança;

	\item Demonstrar uma técnica para transformação de modelos de software em códigos para o projeto;

	\item Propor um método para identificar objetos submersos em ambientes aquáticos com baixa visibilidade, utilizando processamento de imagem;

	\item Propor um método ou técnica para classificar os objetos identificados pelo sistema, visando calcular uma estimativa de sua distância e dimensões;

	\item Projetar um sistema computacional móvel capaz de capturar dados de ambientes aquáticos submersos;

	\item Validar o sistema proposto, pela análise de testes práticos e simulados, a fim de examinar a sua eficácia e aplicabilidade.

\end{enumerate}
    

\section{Organização do trabalho}
A introdução deste trabalho apresentou: o contexto, definição do problema, e objetivos deste trabalho. Os próximos capítulos estão organizados da seguinte forma:

\par
No \autoref{chapter:fund_teo} \textbf{Conceitos e Definições}, são apresentados os conceitos abordados neste trabalho, especificamente: Sistemas Embarcados, Modelos formais para verificação de sistemas, Visão Computacional, \textit{Frameworks} para identificação de imagem.

\par
No \autoref{chapter:correlatos} \textbf{Trabalhos Correlatos}, serão apresentados o método de pesquisa bibliográfica utilizado, sendo ele a revisão sistemática, seguido do resultado encontrado com esta pesquisa e, por fim, a contribuição dos artigos utilizados no desenvolvimento do projeto.
\par
No \autoref{chapter:metodo} \textbf{Método Proposto}, são descritas as etapas de execução do método proposto que consiste na construção de um sistema computacional de suporte a coleta de imagens em ambientes fluviais. Que irá ser apto a detectar objetos e classificá-los.

\par
No \autoref{chapter:consideracoes}, será apresentado o cronograma de atividades a ser realizado na elaboração do método durante o desenvolvimento do projeto.

\par
%No \autoref{chapter:resultados} \textbf{Resultados Preliminares}, descreve-se a execução de uma avaliação experimental sobre os resultados obtidos no desenvolvimento do método proposto, por meio da utilização de \textit{benchmarks} públicos de códigos para testes.

\par
E, por fim, no \autoref{chapter:consideracoes} \textbf{Considerações parciais e trabalhos futuros}, serão apresentas as considerações parciais e as sugestões de trabalhos futuros que podem ser desenvolvidos.